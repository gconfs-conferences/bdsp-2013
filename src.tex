\documentclass[12pt]{beamer}

% \usepackage[sc]{mathpazo}
% \linespread{1.05}
\usepackage[T1]{fontenc}
\usepackage[francais]{babel}

\usepackage{lmodern}
%\usetheme{Szeged}
\usepackage{fontspec}
\usepackage{graphicx}

\usepackage{tikz}
\usetikzlibrary{shapes,arrows}

\tikzstyle{decision} = [diamond, draw, fill=blue!20, 
    text width=4.5em, text badly centered, node distance=3cm, inner sep=0pt]
\tikzstyle{block} = [rectangle, draw, fill=blue!20, 
    text width=5em, text centered, rounded corners, minimum height=4em]
\tikzstyle{line} = [draw, -latex']
\tikzstyle{cloud} = [draw, ellipse,fill=red!20, node distance=3cm,
    minimum height=2em]

% BEGIN STYLE

\setbeameroption{hide notes}
\setbeamertemplate{note page}[plain]

\usetheme{default}
\beamertemplatenavigationsymbolsempty
\hypersetup{pdfpagemode=UseNone}

\usefonttheme{professionalfonts}
\usefonttheme{serif}
\usepackage{fontspec}
\setmainfont{Open Sans}
\setbeamerfont{note page}{family*=pplx,size=\footnotesize}

\definecolor{foreground}{RGB}{249,242,215}
\definecolor{background}{RGB}{21,21,21}
\definecolor{title}{RGB}{255,31,110}
\definecolor{gray}{RGB}{155,155,155}
\definecolor{subtitle}{RGB}{102,255,204}
\definecolor{hilight}{RGB}{102,255,204}
\definecolor{vhilight}{RGB}{255,111,207}

\definecolor{foreground}{RGB}{21,21,21}
\definecolor{background}{RGB}{242,245,227}
\definecolor{title}{RGB}{255,31,110}
\definecolor{gray}{RGB}{155,155,155}
\definecolor{subtitle}{RGB}{102,255,204}
\definecolor{hilight}{RGB}{102,255,204}
\definecolor{vhilight}{RGB}{255,111,207}

\useinnertheme{rectangles}

\setbeamercolor{titlelike}{fg=title}
\setbeamercolor{subtitle}{fg=subtitle}
\setbeamercolor{institute}{fg=gray}
\setbeamercolor{normal text}{fg=foreground,bg=background}
\setbeamercolor{item projected}{bg=foreground, fg=background}
\setbeamercolor{structure}{fg=foreground}
\setbeamercolor{subitem}{fg=gray}
\setbeamercolor{itemize/enumerate subbody}{fg=gray}

\setbeamertemplate{itemize subitem}{{\textendash}}

\setbeamerfont{itemize/enumerate subbody}{size=\footnotesize}
\setbeamerfont{itemize/enumerate subitem}{size=\footnotesize}

\setbeamertemplate{footline}{
    \raisebox{5pt}{\makebox[\paperwidth]{\hfill\makebox[20pt]{\color{gray}
    \scriptsize\insertframenumber}}}\hspace*{5pt}}

\addtobeamertemplate{note page}{\setlength{\parskip}{12pt}}

% END STYLE

\title[Bien démarrer son projet]{Bien démarrer son projet}

\author{
    toogy, Horgix \& Co.
}

\institute{GConfs - EPITA}

\date{15 novembre 2013}

\begin{document}

{
    \setbeamertemplate{footline}{} % no page number here
    \frame{
        \titlepage
    }
}

\section*{Introduction}

\begin{frame}
    \frametitle{Table des matières}
    \tableofcontents[pausesections]
\end{frame}

\section{Organiser son projet}

\begin{frame}
    // TODO: Brainstorming +I
\end{frame}

\begin{frame}
    // TODO: Knife the baby +I
\end{frame}

\begin{frame}
    \begin{tikzpicture}[node distance = 2cm, auto]
        % Place nodes
        \node [block] (init) {initialize model};
        \node [cloud, left of=init] (expert) {expert};
        \node [cloud, right of=init] (system) {system};
        \node [block, below of=init] (identify) {identify candidate models};
        \node [block, below of=identify] (evaluate) {evaluate candidate models};
        \node [block, left of=evaluate, node distance=3cm] (update) {update model};
        \node [decision, below of=evaluate] (decide) {is best candidate better?};
        \node [block, below of=decide, node distance=3cm] (stop) {stop};
        % Draw edges
        \path [line] (init) -- (identify);
        \path [line] (identify) -- (evaluate);
        \path [line] (evaluate) -- (decide);
        \path [line] (decide) -| node [near start] {yes} (update);
        \path [line] (update) |- (identify);
        \path [line] (decide) -- node {no}(stop);
        \path [line,dashed] (expert) -- (init);
        \path [line,dashed] (system) -- (init);
        \path [line,dashed] (system) |- (evaluate);
    \end{tikzpicture}
\end{frame}

\section{Git et le versioning}

\begin{frame}
    // TODO: Slides de la partie
\end{frame}

\section{Introduction à XNA}

\begin{frame}
    \begin{center}
        \vspace{1cm}
        
        {\LARGE Introduction à XNA} \\

        \vspace{0.5cm}

        \includegraphics[scale=0.02]{img/nyancat.png}
    \end{center}
\end{frame}

\section{TP}

\begin{frame}
    // TODO: Slides de la partie
\end{frame}

\end{document}
